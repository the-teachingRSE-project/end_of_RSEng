\documentclass{eceasst}
% This is an empty ECEASST article that can be used as a template
% by authors.

% Required packages
% =================
% Your \usepackage commands go here.
\usepackage{fancybox,framed}
\usepackage{footnote}
\makesavenoteenv[FramedParagraphWithFootnotes]{framed} % for author notes
\usepackage{color}
\usepackage[dvipsnames]{xcolor}
\usepackage[most]{tcolorbox}


\definecolor{easstblue}{rgb}{.05,.32,.66}


\newcounter{whatiscounter}
\newtcbtheorem[use counter=whatiscounter, number format=\Alph]{whatis}{What do we want to have achieved}{
  enhanced,
  segmentation engine=empty,
%  beamer,
fuzzy shadow={2mm}{-1mm}{0mm}{0.1mm}%
{black!50!white},
  sharp corners,
  attach boxed title to top left={
    yshifttext=-1mm
  },
%  halign lower=flush center,
  colback=white,
  colframe=BurntOrange!75!white,
  fonttitle=\bfseries,
  attach boxed title to top left={yshift=-2mm,xshift=3mm},
  fontlower=\itshape,
  boxed title style={
    sharp corners,
    size=small,
    colback=BurntOrange!75!black,
    colframe=BurntOrange!75!black,
  } 
}{ist}

\newcounter{storycount}
\newtcbtheorem[use counter=storycount]{story}{Story}{
  enhanced,
  segmentation engine=empty,
%  beamer,
fuzzy shadow={2mm}{-1mm}{0mm}{0.1mm}%
{black!50!white},
  sharp corners,
  attach boxed title to top right={
    yshifttext=-1mm
  },
%  halign lower=flush center,
  colback=white,
  colframe=easstblue,
  fonttitle=\bfseries,
  attach boxed title to top right={yshift=-2mm,xshift=3mm},
  fontlower=\itshape,
  boxed title style={
    sharp corners,
    size=small,
    colback=easstblue!70!black,
    colframe=easstblue!70!black,
  }
}{str}


\newcommand{\Eg}{E.\,g.}
\newcommand{\eg}{e.\,g.}



% Article frontmatter
% ===================
\title{Future challenges for Research Software Engineering} % Title of the article
%\short{} % Short title of the article (optional)
\newcommand{\authorRef}[1]{\texorpdfstring{\autref{#1}}{}}
\author{
Florian Goth\authorRef{1},
Leyla Jael Castro\authorRef{2},
Gerasimos Chourdakis\authorRef{3},
Simon Christ\authorRef{4},
Jeremy Cohen\authorRef{5},
Jean-Noël Grad\authorRef{6},
Magnus Hagdorn\authorRef{7},
Toby Hodges\authorRef{8},
Jan Linxweiler\authorRef{9},
Frank Löffler\authorRef{10},
Michele Martone\authorRef{11},
Jan Philipp Thiele\authorRef{9}\textsuperscript{,}\authorRef{12},
Nicholas A. Del Grosso\authorRef{13}
} % Authors and references to addresses
\institute{% Institutes with labels
\autlabel{1} Institut für theoretische Physik 1, University of Würzburg, 97074, Würzburg, Germany\par
\autlabel{2} ZB MED Information Centre for Life Sciences, Cologne, Germany\par
\autlabel{3} Institute for Parallel and Distributed Systems, University of Stuttgart, Stuttgart, Germany\par
\autlabel{4} Leibniz University Hannover, Department of Cell Biology and Biophysics, Computational Biology, Germany\par
\autlabel{5} Imperial College London, London, UK\par
\autlabel{6} Institute for Computational Physics, University of Stuttgart, Germany\par
\autlabel{7} Geschäftsbereich IT, Charité Universitätsmedizin Berlin, Germany\par
\autlabel{8} The Carpentries, USA\par
\autlabel{9} Technische Universität Braunschweig, Germany\par
\autlabel{10} Michael Stifel Center Jena {\&} Friedrich Schiller University Jena, Germany\par
\autlabel{11} Leibniz Supercomputing Centre, Garching, Germany\par
\autlabel{12} Weierstrass Institute, Berlin, Germany;
              Leibniz University Hannover, Institute of Applied Mathematics, Scientific Computing, Hannover, Germany\par
\autlabel{13} Institute for Experimental Epileptology and Cognition Research, Uniklinikum Bonn, Germany
}
\abstract{It's 2025 and we are, admittedly, still working on establishing and growing Research Software Engineering (RSE)
as a domain in its own right, with the hope of ensuring that it gets the recognition it deserves.
Nonetheless, as the use of digital tools becomes ever more pervasive in a researcher's day-to-day tasks, there are already people suggesting that RSE will
cease to exist as a separate field, and the specialist skills that RSEs provide to support and undertake research will
become a core part of the tooling and skill set of researchers in almost all other domains.
Are we already approaching ``the end of Research Software Engineering''?! While we believe that it is certainly too early to
already herald the demise of RSE as a separate field,
we also recognise the need to acknowledge that the future is in constant flux and
it is worthwhile discussing how RSE will/could change in response to these upcoming challenges in an open discussion.} % Abstract of the article
\keywords{Keywords go here.} % Keywords for the article

\begin{document}
\maketitle

% Main part of your article
% =========================
\section{Introduction}

Research Software Engineering is still a developing field,
with increasing integration into various aspects of research.
In this opinion paper, we try to imagine the state of our field
and community in the mid-term future, focusing on future challenges.
We present here the results of a workshop discussion at the
deRSE25 conference in Karlsruhe\cite{Goth2025EndRSEng}, without
claiming to represent the entire deRSE community.
Nevertheless, by disseminating the discussed ideas,
we aim to engage the community in creating a worthwhile future
and to advance the conversation about how Research Software Engineering should, or could, develop.
In the following sections, we consider a number of areas:
the evolution of the RSE definition;
the development of RSE education;
the growth of the RSE communities;
the impact of automation and generative artificial intelligence in RSE work;
the emerging ethical considerations;
the transformations expected in research organisations;
and, finally, if there will be other areas in academia covered by new groups of RTPs (\emph{Research Technical Professional}s).
We conclude the paper with a summary and with our wishes and recommendations.

\section{Aspects of this future}
%\subsection{What do we want to have achieved by 2035}
Looking ahead over a 10-year time frame, we hope that the RSE community will have
contributed to making research software and its outputs
more sustainable, robust and maintainable.
We also hope that Research Software becomes an accepted research artefact on an equal footing with data and publications.
We see this happening in a number of ways,
particularly through movements such as DORA~\cite{DORA} and CoARA~\cite{COARA} which are advocating for a change in the way that research is assessed and outputs recognised.
However, we want to ensure that one of the core achievements made by the RSE community in this time
frame is putting the infrastructure and training opportunities in place, alongside the necessary
advocacy, to ensure that researchers receive core software skills as part of their scientific training.
Developing software tools and scripted workflows is becoming an increasingly important part of everyday
research activities.
Therefore, it is vital to Improve digital skills of all researchers.
This will also allow RSEs to focus on other more advanced and specialist
areas. In this context, we do not foresee the demise of RSE as an independent role. However, we
do see the work of RSEs developing and shifting as the research community changes.

The specialist skills that RSEs provide take a significant amount of time to develop and maintain.
Therefore, we do not expect domain researchers to also become experts in a range of
advanced RSE topics. 
At the same time, we do recognise that the application of core software development best practices
to enhance and address problems with research codebases, that can often make up the bulk of
current RSE work, will change. As noted above, these are skills that we
would expect any researcher who writes code to have in 10 years time. 
The rapid development of AI means that researchers are likely to have access to tooling that can assist
them in writing robust, sustainable code, lowering the barrier to developing code, or at least changing the
required skill set to some extent.
They can also be expected to have an understanding of frameworks to support testing, packaging and
deployment of their code as well as integration with central domain specific infrastructures.
Again, all these skills will, we anticipate, be gained as part of training infrastructure that will
provide domain scientists with software development expertise as part of their core scientific training.

So where do RSEs fit into this picture? Even today we expect to see RSEs providing technical skills and input
to research activities in a range of different areas, these include:
high-quality, robust architecture and design for research software;
seamless integration with research and data infrastructures;
development of software for novel and emerging architectures;
application of specialist numerical and statistical methods within software;
optimisation of codes for use on high-performance computing infrastructure;
and green computing - efficient implementations of algorithms/methods and use of hardware

It is already difficult to structure the spectrum of RSE tasks and responsibilities today given the spectrum of research software \cite{hasselbring2024}.
It is even more challenging to structure the aspects that will become relevant for the RSE community in the future.
Nevertheless, we decided to consider:
\begin{itemize}
\item The evolution of the term RSE
\item Development of the RSE community
\item What the impact of AI will be to the work of RSEs
\item The impact of complexity and usability in research software
\item Ethics and social considerations
\item How education is reshaped by the digital transformation
\item Changes to academic institutions
\item Impact of alternate career paths for RSEs and other RTPs
\end{itemize}

\begin{story}{Introduction}{intro}
In order to give a more concrete setting, imagine a world ten years from now.
Not too overwhelmingly far ahead in the future, but distant enough for some changes to have occurred.
Over this time frame Kim\cite{Anzt2021} and Kay\cite{Goth2024},
our fictional RSEs who we have used to illustrate specific scenarios in our earlier work,
were able to ride the waves of their career in Research Software Engineering and are now pondering early retirement on a sun blessed beach.
%MM-NOTE: the above scenario (early retirement) is contrary to all trends in retirement age development of public employees across the "global north".
But the world has moved on.
When they look back, what will they see as problems that have been solved by then?
Which tasks are still relevant, where they can be happy that a younger generation is now taking up the baton and carrying forward their work?
What issues have newly emerged, where they are now just happy to say: “Oh well… my successors will take care of that”.
In order to bring this future to life, these two personas will accompany us through the rest of this paper and we will shed light on aspects
of their future in respective small story boxes.
 \end{story}


\subsection{Definition and Evolution of RSEs}
\begin{whatis}{}{definition-evolution-rse}
The term RSE is established in the minds of university boards and leadership, as a valuable member in the academic framework.
New RSEs are trained through a specialized Master's programme, and their support is valued by researchers. We still teach researchers basic software skills, but the content of this lecture has changed drastically.
\end{whatis}
% Here we consider how the topics of RSEng will change in the face of the digitalisation of society.
% Also here: What will our topics be after version control doesn't need to be taught anymore?
% Digitalisation will also feature more heavily in the domain curricula.
In this discussions the workshop participants found that
the digital skills of domain scientists will increase,
and software will become easier to interact with,
reducing reliance on RSEs to train newcomers on the most basic tasks.
This means, that today's workshop content of proper versioning and data management could become obsolete
by using more intuitive tools.
Introductory courses can then focus on more higher-level concepts.
User training sessions might focus on more advanced topics
that cater to a smaller audience, or involve mentoring more frequently.

While the RSE core competencies\cite{Goth2024} might remain unchanged,
there might be a shift in the responsibilities of RSEs,
in the form of a partial reallocation of the training budget
to the other core competencies.
Regarding the definition of RSEs, we will still have a spectrum of competencies,
from full professional RSEs to domain scientists who develop research software.

What will \emph{not} change over that time frame is the RSE's ability to remain
flexible and embrace change, and keep an eye on the wider landscape,
picking up new tools and skills as needed.
Communication and collaboration skills will remain essential to engage
with domain scientists and help them find technical solutions.\\

Looking at that on a broader scale, especially the education aspect deserves some attention.\\
The Covid-19 pandemic acted as a catalyst for the digital transformation
of higher education\cite{Bygstad2022}. Students and early-career researchers
are now more familiar with digital learning tools, which lead to a shift
in the responsibilities of teachers. To quote from the original study,
``with so many digital resources at hand,
the task of the lecturer will be fewer lectures,
and to act more as a facilitator of resources,
and to monitor activities and results over time.''\cite{Bygstad2022}

RSE-relevant open educational resources and Master's programmes are tracked
in the Learning and Teaching RSE database\footnote{\url{https://de-rse.org/learn-and-teach/}}
and UK SSI resources hub\footnote{\url{https://www.software.ac.uk/resource-hub}}.
The Carpentries now offer specialised software workshops
for data scientists\footnote{\url{https://datacarpentry.org}},
librarians\footnote{\url{https://librarycarpentry.org}},
and soon HPC practitioners\footnote{\url{https://www.hpc-carpentry.org}}.

\begin{story}{The Definition of RSEs}{defrse}
Kim and Kay chuckled when they looked at their original job descriptions from 2020: "build tools for scientists, manage data, write reusable code."
Now, their roles are unrecognizable: half AI whisperer, half research ethicist, and part-time diplomat negotiating between synthetic and human collaborators.
They now not only manage software, but also the personalities of autonomous lab agents and negotiate data-sharing agreements with international AI consortia.
Despite the tech upheaval, their deep knowledge of scientific workflows and the messy, human way science actually gets done remains indispensable.
When labs faltered under algorithmic bias or unclear model provenance, it was Kim and Kay who transformed the chaos back into meaningful order.
They understand that science does not move in straight lines, and that trust between collaborators, institutions, and even machines has to be built, not coded.
Their job titles have changed, but their role as bridges between logic and life remains the same.
In a future of shifting code and constant reinvention, it is their understanding of people that never went out of date.

% Kim and Kay started out as Research Software Engineers.
% When they began, their work meant cleaning up code, making research reproducible, and helping academics run experiments without crashing HPC clusters. Now, they built systems, where AI generated hypotheses, designed experiments, and even debated its own conclusions. Their role had shifted from supporting research to shaping how knowledge was created, validated, and understood—an epistemological responsibility as much as a technical one. “We used to debug scripts,” Kay said, watching a model suggest a reality-bending theory of space-time. Kim nodded: “Now we debug truth.”
\end{story}

\subsection{Development of the RSE community}
\begin{whatis}{}{development-rse-community}
National RSE communities are increasingly collaborating with local partners, fostering stronger networks across countries. Regular international conferences further amplify these connections, promoting the global exchange of ideas. The introduction of RSE Master's programmes, coupled with a shift in culture towards valuing Research Software, has significantly elevated its profile. This growing recognition has led to a surge in membership of national societies, reflecting the increasing importance of RSEng in the research landscape.
\end{whatis}
In recent years, we have seen the formation of RSE institutes and advocacy groups, such as
the UK Software Sustainability Institute\footnote{\url{https://www.software.ac.uk}},
the US Research Software Sustainability Institute\footnote{\url{https://urssi.us}},
the European Virtual Institute for Research Software Excellence\footnote{\url{https://everse.software}},
and the EOSC Opportunity Area Expert Group 7 on Research Software%
\footnote{\url{https://eosc.eu/opportunity-area-exp/oa7-research-software}}.
We expect these initiatives to help increase the visibility and accelerate
the professionalisation%
of the RSE role.
In addition, a multitude of RSE communities self-organised at the national
and regional level to support more viable RSE career paths in academia.
We can expect RSE positions to become more easily accessible, possibly through centrally-funded long-term positions at research institutions.
The community of RSE's will get a significant boost in their acceptance if RSE Master's programmes
become available at various universities.
With increased numbers, the visibility of the RSE community to industry will be enhanced.
\begin{story}{Development of RSE community}{RSEcommunity}
Kay stands proudly at the podium of the first International Research Software Engineering Conference held in the Southern Hemisphere, a historic moment for both her career and the global RSE community. Having moved to South Africa, she speaks passionately about the unique challenges researchers in the Global South face: limited access to high-performance computing, unreliable funding, and the persistent digital divide. Her keynote highlights how local innovation thrives despite constraints, often out of necessity rather than choice. The audience, a diverse mix of global experts, listens intently as she calls for equitable collaborations, not just inclusion as an afterthought.
\end{story}

\subsection{Impact of AI and automation}
\begin{whatis}{}{impact-ai-automation}
RSEs understand how AI models work, and can help researchers reap their benefits, while
clearly knowing their limits.
\end{whatis}
AI is already transforming research processes by enabling rapid analysis of massive datasets, uncovering patterns and insights that would be impossible or prohibitively time-consuming for humans alone to find.
In research and development, AI accelerates scientific breakthroughs in fields like healthcare, climate science, and materials engineering by automating complex modelling and predictive tasks.
For software development, AI-powered tools automate code generation, bug detection, and testing, significantly increasing productivity and reducing human error.
Developers are shifting from manual coding to orchestrating AI-driven workflows, focusing more on high-level design, integration, and creative problem-solving while AI handles repetitive or routine tasks.
The integration of AI is democratizing access to advanced technologies and analytics, enabling non-experts to participate in research and software creation, and fostering greater collaboration across disciplines.
As AI continues to evolve, it will further blur the lines between human and machine roles, making research and software development faster, more efficient, and more innovative than ever before.
With the general availability of AI the interaction with researchers will change at a fundamental level.
Whereas previously RSEs have been points of knowledge that could translate human language formulated problems into technical descriptions and then provide solutions to them,
it can be foreseen that AI will drive this low-key interaction to a highly specialized level of knowledge,
since AI copilots are available 24/7.
Since AI systems are so democratically available and will be used as black boxes,
the interpretation of their outputs will be far more important.
Currently we are expanding and documenting our work in the open and making our code available,
therefore there will be more structured material to train GenAI on.
However, as people get quicker answers from GenAI, the community resources (\eg Stack Overflow, Wikipedia)
get poorer.

\begin{story}{The interaction of RSEs with AI}{RSEs_with_AI}
Kim develops advanced scientific software in tandem with an AI agent that codes, debugs, and simulates experiments alongside him. The AI proactively suggests optimisations, generates reproducible workflows, and adapts to Kim’s coding style and domain-specific needs. With this partnership, Kim accelerates his research output while maintaining full control over scientific direction and ethical oversight.
\end{story}

\subsection{Impact of Complexity and Usability in Research Software}
\begin{whatis}{}{abstraction}
With a palette of readily reusable software components,
and with their skills in managing hierarchical systems of complex software components with clear interfaces,
RSEs can now build reliable, maintainable systems of a scale and capability not previously possible.
\end{whatis}

The workshop participants found that, in earlier learning stages,
training may abstract away complexities that will always be present in research.
For this reason, the participants fear that the capability of researchers to deal with complexity
of software frameworks and libraries will reduce over time.
As a result, RSEs will need to invest more effort in making the interfaces of individual research software components
less complex, as the user experience expectations of researchers will continue to grow.
Participants expect an over-reliance on generative AI tools as an extreme manifestation of this issue.
No concrete examples were given on which complexities are now hidden behind abstractions.

We acknowledge but would also like to challenge the observations of the participants.
We believe that ``we are standing on the shoulders of giants'':
By reducing the complexity of individual software components,
we are now able to develop systems of complexity and capability not previously possible.
This divide-and-conquer approach allows the scientific community to address questions not previously within their grasp.
When it comes to teaching, it is true that students need to learn how to handle complexity,
by being able to develop deep understanding of individual components.
However, it is also important to learn how to manage complexity
by building hierarchical systems of well-encapsulated, reusable components.
It is clear that we need both RSEs that can dive deep into single projects of significant complexity,
as well as RSEs that can manage complex systems of almost black-box components,
and the future of RSE education should aim to prepare students for both roles.

\begin{story}{The benefits of abstraction}{abstraction}
Years ago, Kim had to distribute the software library libkim as a source code package with a custom build system and detailed instructions on how to build it on some common systems,
investing significant time in developing and maintaining this auxiliary toolkit and documentation.
Around the same time, Kay was developing the research analysis tool kay-bio,
which however only provided an application programming interface that
required writing complex code even for simple and common analysis tasks.
Nowadays, Kim could have started directly from a template based on established toolkits and workflows,
while kay-bio accepts configuration in a simpler, standardized format and generative AI assistants
can guide new users in preparing and debugging their analyses.
Meanwhile, new RSEs can very easily integrate libkim, kay-bio, and a multitude
of other tools from the research software corpus into a powerful analysis
platform that allows researchers to get deeper insights from their research data
in just a few clicks.
\end{story}

\subsection{Ethics and social considerations}
\begin{whatis}{}{ethics}
The principles of Good Scientific Practice \cite{dfg_gsp} have been a cornerstone in the conduct of research.
But their relevance hinges on constant updates in light of how research is performed.
By 2035 we want to have achieved accepted rules on the use of AI in research tasks and software development.
Open Science and Open Research are well-established in academia, and supported by respective Open Research Software
strategies of the political actors. We also hope that Data Awareness and Data Protection are still values
that are protected.
\end{whatis}

The ethics and values of the Research Software Engineer are grounded in their practice.
They commit to the health, safety and welfare of the public and act in the interest of society, their employer and their clients \cite{Goth2024}.

% institutional bonds and ethics
Sometimes these values may conflict with the aims of their peers or institution. For example, their PI might insist on quickly developing some software without adhering to current best practices.
A code of ethics and professional conduct for RSEs should be developed by institutions together with RSEs themselves.
This can then form the basis for institutional policies and guidelines.

% society implications
The work being undertaken by RSEs may well have direct, or indirect,
ethical implications in areas such as social engineering \cite{s2erc,siadati2024},
climate \cite{Lannelongue2023} and more.
Another example might be that they are developing code that can also be used for military purposes, such as numerical solvers or computer vision systems.
Like engineers, RSEs take the responsibility for the software they develop.
RSEs need to be aware of these issues and develop their own ``red lines''.

% data ethics
Similarly to how data collected by scientists should adhere to the FAIR principles \cite{FAIR}, research software developed by RSEs should adhere to the FAIR4RS principles~\cite{FAIR4RS}.

A code of ethics and professional conduct for RSEs should be developed by institutions together with RSEs themselves.
This can then form the basis for institutional policies and guidelines as well as serve as a background for individual developers to consider how software can be used.

\begin{story}{The greater world of Kim and Kay}{ethics}
10 years have passed since Kim and Kay started on their journey of being RSEs. Back then,
at the universities of Eden and Utopia, life was more orderly and harmonious. Kay is now working on a project that develops
software for cheap water monitoring devices in the Global South. Kim is currently designing the software service stack
at the university, and increasingly struggles to find compelling Open Source alternatives that enables them keep complete control of their data
rather than relying on software and tools that are backed by large corporate entities. So far she still has the support of the university board, since it values
strategic autonomy.
Her son Kyle on the other hand started out with a degree in Physics and then took one of the newly created RSE Master's courses,
but quickly decided that work in academia with its time limited contracts is not attractive.
Governments around the world are struggling to reposition themselves in this emerging de-globalised world with its new blocks
and therefore have de-prioritised publicly funded research. Kyle decided to work for a defence company since they offer
more secure employment in this polarized world.
\end{story}

\subsection{How have academic Institutions transformed by then?}
\begin{whatis}{}{academic}
We expect research software to be a central part of scientific practice on the same level as data and publications.
This will be supported by the first generations of RSE-Master graduates that find their way into research departments, and accelerate the pace of science.
Similar to RSEs, that are widely known and understood by university boards in 2035, we expect the concept of RSE units to be in the minds
of academic staff at any level.
We hope to see that universities use the presence of an RSE Department actively as a means to differentiate themselves from other universities.
We hope that the central RSE units have local RSE representatives, or even small RSE groups that handle tasks within a department to oversee specific projects \cite{Kempf2025-draft}.
We also hope, that specialized clusters form that pool the skills of certain specialised RSEs, with certain RSE units specializing in certain tasks like data management, HPC, or visualisation.
At universities which have them, we want to see that researchers naturally accept them as part of their available services.
Additionally researchers will seamlessly interact with infrastructures for using research software.
\end{whatis}
Academia, while claiming to be innovative \cite{wzvgbayern}, has organisationally often adopted time-scales for change that are more reminiscent to institutions thinking on the time scale of generations.
Nevertheless, the presence, even today, of initiatives like \emph{the Hidden Ref} \cite{hiddenref} hints at the potential for specialisation of research tasks. This gives rise to new job descriptions within the academic system such as specialised RSEs or other RTPs.
We have already seen some specialisation trends in recent years such as
the HPC-RSE and AI-RSE which are quite recent developments, whose appeal is stimulated by large public and private investments in HPC facilities and AI technologies.
But we could imagine new RSE specialisations emerging in the fields
of quantum computing, such as the Quantum-RSE, and in federated online platforms for research.
% European AI factories: https://digital-strategy.ec.europa.eu/en/policies/ai-factories
% European federation platform for HPC: https://eurohpc-ju.europa.eu/paving-way-eurohpc-federation-platform-2024-12-19_en
% European quantum computing: https://digital-strategy.ec.europa.eu/en/policies/quantum

On the other hand the existence of people with a certain specialised skillset opens up possibilities for pooling.
Universities will benefit from pooling these specialists into RSE units, and offer their services to researchers.
Hopefully, these RSE-units are endowed with enough trust by the university boards that translates into stable funding.
With longer-term funding and more trust to pick up longer term projects,
some RSEs can focus on code modernisation, clean-up, and modularisation,
which can pay off in faster and easier development in the future.
This is an opportunity that would not be possible if relying on short-term funded solo RSEs.

\emph{Which challenges will still persist}\\
The workshop participants expect that there will still be budgeting and administrative hurdles to creating
permanent and well-funded RSE teams.
Here it can only be strongly hoped, that, with the recognized importance of research software, research councils and funders
support the development with a reliable stream of funding.
This would also require a change of culture in research about the acceptance of research software.\\
While we hope that the barrier to researchers diminishes, the workshop participants
still expect some work to do. We hope that the compartmentalisation that occurs will help researchers re-prioritise their workload.
An upcoming challenge will be that, with the emergence of more dedicated RSE units,
networking between them will come into the foreground, because there should be a strong need to share best practices
among institutions with successful RSE Teams.
On a more organisational level, we expect the sharing of models on how to organise different RSEs with different areas of specialist expertise, to fit the different organisational constraints.
After that, organisations will have to come up with updated models for hiring people.
While a standardized RSE Master's degree makes it easier to develop the next generation of practitioners with the necessary technical skills, it will be interesting to see if
RSE are still often hired by word of mouth within an organisation(``I have this PhD here, before their PostDoc they would be a good fit for the RSE department...'' ) or if the RSE Master's degree ensures a hiring process on a more standardised level.

\begin{story}{Changes to the Universities}{changes}
Kim found work years ago at one of the first RSE units in Arcadia. After a couple of years he moved to the university of
Avalon which was interested in setting up their own department for RSEs in order to not fall behind the university of Arcadia.
Kim builds upon the model set forth by the University of Arcadia based on his valuable experience. Since the university
of Avalon has to follow different rules, not everything can be straightforwardly adapted. Life would be so much easier
if at least university regulations would be homogenised, no?
\end{story}

\emph{New challenges}
If we follow the ideas laid out in \cite{Kempf2025-draft}, then in addition to university wide RSE units,
there will be embedded RSEs in the departments. Shaping and developing these network bonds will be a task for the future.
While networking embedded RSEs and RSE units is essential for building robust, sustainable research software infrastructure, the question arises whether every group truly needs to host and manage its own infrastructure.
When each team or institution hosts its own systems, this often leads to siloed environments where resources, expertise, and data are isolated, limiting collaboration and creating inefficiencies. Such silos can result in duplicated efforts, higher costs, and challenges in maintaining data integrity and governance, as each group independently manages similar infrastructure and workflows. While local control offers flexibility, it also increases complexity and makes it harder to share innovations or best practices across the wider research community. To overcome these challenges, federated infrastructure models are needed, enabling shared access, interoperability, and coordinated development while still allowing for some local autonomy \cite{Jelinek2021}.
Federated approaches break down silos, promote knowledge exchange, and ensure that infrastructure investments benefit the broader ecosystem rather than just isolated groups.

\subsection{Impact of Alternate Career Paths for Researchers on the Structure of Research Institutes}
\begin{whatis}{}{career-paths}
We would like to see the emergence of more dedicated personnel, i.e. other (digital) RTPs next to the RSE.
Also we hope that researchers can build careers at the interface of academia, industry, and policy (\eg in think tanks, NGOs,
or organisations that do knowledge transfer).
\end{whatis}
RSEs can either be embedded in research groups (belonging to the mid-level university staff),
or be members of larger RSE units.\cite{Kempf2025-draft}
The participants had the opinion that focusing on embedded RSEs would make onboarding new RSEs to existing projects easier.
This, however, comes with additional funding challenges,
as long-term funding can more difficult to acquire in lower tiers of an of a research organisation (i.e. at departmental or individual research group level).
Third-party funding can mitigate this issue in the mid-term.
However, the participants considered central RSE groups more likely in the future,
due to the fact that there are already examples and activities from various universities in this direction
(examples include the universities of Jena, Heidelberg and Göttingen, as well as several universities in the UK where the concept of dedicated RSE groups began to emerge around 2013).
We expect funding in central groups to evolve from initially temporary positions to permanent positions,
as the groups prove themselves to be increasingly valuable and, hence, sustainable.
Embedded RSEs and central groups will need to network with each other,
and potentially request and coordinate support from external RSEs.

\begin{story}{New career paths}{New career paths}
Ken, though trained as a physicist, found his passion not in theory, but in keeping the quantum computers humming. He became the go-to expert for troubleshooting entangled qubits and stabilizing delicate cryogenic systems. His unique blend of physics knowledge and hands-on problem solving earned him a new title among his peers: the first Quantum RTP. In a world where theory meets reality, Ken quietly ensures the future of quantum computing stays on track.
\end{story}
\section{What is in this future for RSEs and where to go from here?}
%FG-NOTE: I think we can use this part to wrap things a bit up, and describe how this future then looks like.
%GC-NOTE: Shall we add another story box here? Ideas what we write here?
It is hard to conclude a piece written about the future, but we have tried to list some of the trends that we foresee.
The material presented here is based on the outputs of a workshop, but extended with the authors' ideas.
Nevertheless there are some things that we declared out of scope here.
Among that is the ongoing digital transformation in wider society.
Digital competencies in the general population continue to increase, making jumping to software development easier than before.
But this digitalisation will also lead to changes in the way societies interact and therefore will translate into changes in how academia perceives itself.
We have also not delved into the interaction with industry much.
There are also some trends that we believe will stay with us over the coming years.
The differentiation of job profiles in academia, that has started with profiles such as the RSE will continue.
Societies will be facing new questions, that arise from technologies such as generally available AI systems.

We hope that the standard of research software will have risen across the board, enabling easier reuse of prior software outputs in new research.
We also hope that a cultural change is taking place to recognize research software as an integral part of science that it is
widely recognized and rewarded across academia and is supported by political actors.
To conclude, RSEs will very likely be doing different work
to what they are doing today, but their core task of enhancing the scientific output of researchers -- other human
beings -- will stay the same.
They will still develop software artefacts together with scientists,
since better software leads to better research,
and better research leads to better software.

\begin{acknowledge}
% MM-NOTE: I'd keep the following sentence out, out of several reasons.
% AI Systems have been harmed in the creation of this work.
We thank all the participants of our workshop at deRSE25 in Karlsruhe for their valuable input!  JC acknowledges support from the UKRI-EPSRC-funded STEP-UP project under grant EP/Y530608/1.
\end{acknowledge}

% Bibliography with BibTeX
% ========================
\bibliographystyle{eceasst}
\bibliography{bibliography/bibliography,extra}

\end{document}
