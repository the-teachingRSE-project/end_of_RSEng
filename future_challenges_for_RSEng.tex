\documentclass{eceasst}
% This is an empty ECEASST article that can be used as a template
% by authors.

% Required packages
% =================
% Your \usepackage commands go here.
\usepackage{fancybox,framed}
\usepackage{footnote}
\makesavenoteenv[FramedParagraphWithFootnotes]{framed} % for author notes
\usepackage{color}
\usepackage[dvipsnames]{xcolor}
\usepackage[most]{tcolorbox}


\definecolor{easstblue}{rgb}{.05,.32,.66}


\newcounter{whatiscounter}
\newtcbtheorem[use counter=whatiscounter, number format=\Alph]{whatis}{What is 2035}{
  enhanced,
  segmentation engine=empty,
%  beamer,
fuzzy shadow={2mm}{-1mm}{0mm}{0.1mm}%
{black!50!white},
  sharp corners,
  attach boxed title to top left={
    yshifttext=-1mm
  },
%  halign lower=flush center,
  colback=white,
  colframe=BurntOrange!75!white,
  fonttitle=\bfseries,
  attach boxed title to top left={yshift=-2mm,xshift=3mm},
  fontlower=\itshape,
  boxed title style={
    sharp corners,
    size=small,
    colback=BurntOrange!75!black,
    colframe=BurntOrange!75!black,
  } 
}{ist}

\newcounter{storycount}
\newtcbtheorem[use counter=storycount]{story}{Story}{
  enhanced,
  segmentation engine=empty,
%  beamer,
fuzzy shadow={2mm}{-1mm}{0mm}{0.1mm}%
{black!50!white},
  sharp corners,
  attach boxed title to top right={
    yshifttext=-1mm
  },
%  halign lower=flush center,
  colback=white,
  colframe=easstblue,
  fonttitle=\bfseries,
  attach boxed title to top right={yshift=-2mm,xshift=3mm},
  fontlower=\itshape,
  boxed title style={
    sharp corners,
    size=small,
    colback=easstblue!70!black,
    colframe=easstblue!70!black,
  }
}{str}






% Article frontmatter
% ===================
\title{Future challenges for Research Software Engineering} % Title of the article
%\short{} % Short title of the article (optional)
\author{
Florian Goth\texorpdfstring{\autref{1}}{},
Leyla Jael Castro\texorpdfstring{\autref{1}}{},
Gerasimos Chourdakis\texorpdfstring{\autref{1}}{},
Simon Christ\texorpdfstring{\autref{1}}{},
Jeremy Cohen\texorpdfstring{\autref{1}}{},
Jean-Noël Grad\texorpdfstring{\autref{1}}{},
Magnus Hagdorn\texorpdfstring{\autref{1}}{},
Toby Hodges\texorpdfstring{\autref{1}}{},
Jan Linxweiler\texorpdfstring{\autref{1}}{},
Frank Löffler\texorpdfstring{\autref{1}}{},
Michele Martone\texorpdfstring{\autref{1}}{},
Jan Philipp Thiele\texorpdfstring{\autref{1}}{},
Nick del Grosso\texorpdfstring{\autref{1}}{}
} % Authors and references to addresses
\institute{\autlabel{1} Fantasy University} % Institutes with labels
\abstract{It's 2025 and we are, admittedly, still working on establishing and growing Research Software Engineering (RSE)
as a domain in its own right, with the hope of ensuring that it gets the recognition it deserves.
Nonetheless, as the use of digital tools becomes ever more pervasive in a researcher's day-to-day tasks, there are already people suggesting that RSE will
cease to exist as a separate field, and the specialist skills that RSEs provide to support and undertake research will
become a core part of the tooling and skill set of researchers in almost all other domains.
+Are we already approaching ``the end of Research Software Engineering''?! While we believe that it is certainly too early to
already herald the demise of RSE as a separate field,
we also recognise the need to acknowledge that the future is in constant flux and
it is worthwhile discussing how RSE will/could change in response to these upcoming challenges in an open discussion.} % Abstract of the article
\keywords{Keywords go here.} % Keywords for the article

\begin{document}
\maketitle

% Main part of your article
% =========================
\section{Introduction}
This opinion article stems from a Bof session at deRSE25 in Karlsruhe\cite{Goth2025EndRSEng}.
Therefore it stems from a discussion among the participants of this workshop and hence cannot claim that it represents the entire deRSE community.
Nevertheless we hope, that we succeeded in writing up something that
\begin{itemize}
\item is pleasant to read,
\item gives something to think about,
\item motivates engaging in work towards a better future.
\end{itemize}
In order to structure the discussion, we decided on an imaginable setting ten years from now.
In this world we will consider the dimensions of 
\begin{itemize}
\item Evolution of the term RSEs
\item The development of academic institutions
\item What will the impact of AI be to the work of RSEs?
\item How will academic education have transformed by then
\item Ethics and social considerations
\end{itemize}
In each aspect we will quickly summarise on what we want to have achieved by then, and what will be the challenges that arise from that.



\section{The Situation}
In order to give a more explicit setting, imagine a world ten years from now.
Not too much to be overwhelmingly far in the future, but distant enough for some changes to occur.
Over this timeframe Kim\cite{Anzt2021} and Kay\cite{Goth2024}
were able to ride the waves of their career in Research Software Engineering and are now pondering early retirement on a sunset blessed beach.
%MM-NOTE: the above scenario (early retirement) is contrary to all trends in retirement age development of public employees across the "global north".
But the world has moved on.
When they look back, what will they see as problems which are now solved?
Which tasks are persisting, where they are now happy that a younger generation is now taking up the baton and carries forward their work.
What issues have newly emerged, where they are now just happy to say: “Oh well… My successor takes care of that”.
\begin{story}{Introduction}{intro}
In order to bring this future to life these two personas will accompany us and we will shed light on certain aspects
of their future in respective small story boxes.
 \end{story}

\section{Aspects of this future}
It is difficult to structure the different aspects in describing the future. We decided to consider
\begin{itemize}
\item The evolution of the term RSE,
\item How education is reshaped by the digital transformation,
\item What the impact of AI to the work of RSEs will look like,
\item Changes to the Academic Institutions,
\item Ethics and social considerations,
\item Impact of alternate career paths for RTPs (\emph{Research Technical Professional}s).
\end{itemize}

%\subsection{What do we want to have achieved by 2035}

Looking ahead over a 10-year timeframe, we hope that the RSE community will have
contributed to make research software and its outputs
more sustainable, robust and maintainable. We see this happening in a number of ways. However,
we want to ensure that one of the core achievements made by the RSE community in this time
frame is putting the infrastructure and training opportunities in place, alongside the necessary
advocacy, to ensure that researchers receive core software skills as part of their scientific training.

With development of software tools and scripted workflows becoming an increasingly important part of everyday
research activities, this is vital and it will allow RSEs to focus on other more advanced and specialist
% MM-NOTE: if "this is vital" is referring to "software skills" it's not clear -- different paragraphs
areas. In this context, we do not foresee the demise of RSE as an independent role. However, we
do see the work of RSEs developing and shifting as the research community changes.

The specialist skills that RSEs provide take a significant amount of time to develop and maintain.
We would, therefore, not expect domain researchers to also become experts in a range of
advanced RSE topics. 
At the same time, we do recognise that the application of core software development best practices
to enhance and address problems with research codebases, that can often make up the bulk of
current RSE work, will change. As noted above, these are skills that we
would expect any researcher who writes code to have in 10 years time. 
The rapid development of AI means that researchers are likely to have access to tooling that can assist
them in writing robust, sustainable code, lowering the barrier to developing code, or at least changing the
required skill set to some extent.
They can also be expected to have an understanding of frameworks to support testing, packaging and
deployment of their code as well as integration with central domain specific infrastructures.
Again, all these skills will, we anticipate, be gained as part of training infrastructure that will
provide domain scientists with software development expertise as part of their core scientific training.

So where do RSEs fit into this picture? We expect to see RSEs providing technical skills and input
to research activities in a range of different areas, including the following:
\begin{itemize}
  \item High-quality, robust architecture and design for research software
  \item Seamless integration with research and data infrastructures
  \item Development of software for novel and emerging architectures
  \item Application of specialist numerical and statistical methods within software
  \item Optimisation of codes for use on high-performance computing infrastructure
  \item Green computing - efficient implementations of algorithms/methods and use of hardware
  \item ...
\end{itemize}

\subsection{Definition and Evolution of RSEs}
\begin{framed}
\hfill How will the definition of an RSE evolve in ten years?\\

\hfill What will not change in the essence of an RSE ten years from now?
\end{framed}
% Here we consider how the topics of RSEng will change in the face of the digitalization of society.
% Also here: What will our topics be after version control doesn't need to be taught anymore?
% Digitalisation will also feature more heavily in the domain curricula.

The digital skills of domain scientists will increase,
and software will become easier to interact with,
reducing reliance on RSEs to train newcomers on the most basic tasks
(JN: for example, versioning and archiving?).
User training sessions might focus on more advanced topics
that cater to a smaller audience, or involve mentoring more frequently.

Regarding the definition of RSEs, we will still have a spectrum of competencies,
from full professional RSEs to domain scientists who develop research software.
While the RSE core competencies\cite{Goth2024} might remain unchanged,
there might be a shift in the responsibilities of RSEs,
in the form of a partial reallocation of the training budget
to the other core competencies.

What will \emph{not} change over that timeframe is the RSE's ability to remain
flexible and embrace change. Keep an eye open and pick up tools and skills
as needed as we go along.
Communication and collaboration skills will remain essential to engage
with domain scientists and help them find technical solutions.

\begin{FramedParagraphWithFootnotes}
JN-NOTE:
The Covid-19 pandemic acted as a catalyst for the digital transformation
of higher education\cite{Bygstad2022}. Students and early-career researchers
are now more familiar with digital learning tools, which lead to a shift
in the responsibilities of teachers. To quote from the original study,
``with so many digital resources at hand,
the task of the lecturer will be fewer lectures,
and to act more as a facilitator of resources,
and to monitor activities and results over time.''\cite{Bygstad2022}

RSE-relevant open educational resources and master's programmes are tracked
in the Learning and Teaching RSE database\footnote{\url{https://de-rse.org/learn-and-teach/}}
and UK SSI resources hub\footnote{\url{https://www.software.ac.uk/resource-hub}}.
The Carpentries now offer specialised software workshops
for data scientists\footnote{\url{https://datacarpentry.org}},
librarians\footnote{\url{https://librarycarpentry.org}},
and soon HPC practitioners\footnote{\url{https://www.hpc-carpentry.org}}.
\end{FramedParagraphWithFootnotes}

\begin{FramedParagraphWithFootnotes}
JN-NOTE:
In recent years, we saw the formation of RSE institutes and advocacy groups, such as
the UK Software Sustainability Institute\footnote{\url{https://www.software.ac.uk}},
the US Research Software Sustainability Institute\footnote{\url{https://urssi.us}},
the European Virtual Institute for Research Software Excellence\footnote{\url{https://everse.software}},
and the EOSC Opportunity Area Expert Group 7 on Research Software%
\footnote{\url{https://eosc.eu/opportunity-area-exp/oa7-research-software}}.
We expect these initiatives to help increase the visibility and accelerate
the professionalisation%
\footnote{MM: do we want to decide for UK or US English? Or keep it mixed? Not being a native English speaker I don't know how it reads.}
of the RSE role.
In addition, a multitude of RSE communities self-organised at the national
and regional level to support more viable RSE career paths in academia.
We can expect RSEs to become more easily accessible,
possibly through centrally-funded long-term positions at research institutions.
\end{FramedParagraphWithFootnotes}

\begin{FramedParagraphWithFootnotes}
JN-NOTE:
New RSE specialisations might arise.
HPC-RSE and AI-RSE are quite recent developments,
whose appeal is stimulated by large public and private
investments in HPC facilities and AI technologies.
We could imagine new RSE specialisations to emerge in the fields
of quantum computing and federated online platforms.
% European AI factories: https://digital-strategy.ec.europa.eu/en/policies/ai-factories
% European federation platform for HPC: https://eurohpc-ju.europa.eu/paving-way-eurohpc-federation-platform-2024-12-19_en
% European quantum computing: https://digital-strategy.ec.europa.eu/en/policies/quantum
\end{FramedParagraphWithFootnotes}


\subsection{Education, and changes in the educational system}
FG-NOTE: I wonder if we need a separation between education of RSEs and changes in the education of the broader society.

\subsection{Impact of AI and automation}
Is it better than cheap PhDs?
FG-NOTE: AI will change the interaction with researchers at a fundamental level. Whereas previously RSEs have been points of knowledge that could translate human language formulated problems into technical descriptions and then provide solutions to it,
I foresee that AI drives this low-key interaction to a highly specialized level of knowledge, since the availability of AI is far better.

\subsection{Ethics and social consideration}

The ethics and values of the Research Software Engineer are grounded in their practice.
They commit to the health, safety and welfare of the public and act in the interest of society, their employer and their clients \cite{Goth2023}.

% institutional bonds and ethics
Sometimes these values are in direct conflict with the aims of their institution. For example, their PI might insist on quickly developing some software without adhering to current best practices.
A code of ethics and professional conduct for RSEs should be developed by institutions together with RSEs themselves. This can then form the basis for institutional policies and guidelines.

% society implications
The technology people working as RSE might have, well directly or indirectly, ethical implications in areas such as social engineering [TODO find source], climate [TODO find sourc] and more.
Another example might be that they are developing code that can also be used for military purposes, such as numerical solvers or computer vision systems.
Like engineers, RSEs take the responsibility for the software they develop.
RSEs need to be aware of these issues and develop their own ``red lines''.

% data ethics
Similarly to how data collected by scientists should adhere to the FAIR principles \cite{FAIR}, research software developed by RSEs should adhere to the FAIR4RS principles~\cite{FAIR4RS}.

A code of ethics and professional conduct for RSEs should be developed by institutions together with RSEs themselves.
This can then form the basis for institutional policies and guidelines.

MM-NOTE: (Flo believes it was from Michele) Are we assuming that progress will bring prosperity? Will we remain constrained by needing to produce products/artifacts?

  - Importance of having personal “red lines” and awareness of how software can be used.

\begin{story}{The greater world of Kim and Kay}{ethics}
10 years have passed since Kim and Kay have started on their journey of being RSEs. Back then,
at the universities of Eden and Utopia, life was more orderly and harmonious. Kay is working in a project that develops
software for cheap water monitoring devices in the global south. Kim is currently designing the software service stack
at her university, and increasingly struggles to find compelling Open Source alternatives that lets them keep their data
to software that is backed by large corporate entities. So far she still has the support of her university board, since it values
strategic autonomy.
Her son Kyle on the other hand started out with a degree in Physics and then took one of the newly created RSE Master's courses,
but quickly decided that work in academia with its time limited contracts is not attractive.
Governments around the world are struggling to reposition themselves in this emerging deglobalized world with its new blocks
and therefore have deprioritized publicly funded research. Kyle decided to go to a defense company since they offer a
more secure employment in this polarized world.
\end{story}

\subsection{How have academic Institutions transformed by then?}
\begin{whatis}{What do we want to have achieved}{academic}
We hope to see that universities use the presence of an RSE Department actively as a means to differentiate themselves from other universities.
We hope that the central RSE units have local RSE representatives, or even small RSE groups that handle tasks within a department to oversee specific projects[FIXME: cite group paper].
We also hope, that specialized clusters form, with certain RSE units specializing in certain tasks like data management, HPC, or visualization.
At universities which have them, we want to see that researchers naturally accept them as part of their used services.
\end{whatis}

\subsubsection{Which challenges will still persist}
The workshop participants expect that there will still be budgeting and administrative hurdles to creating
permanent and well-funded RSE teams.
While we hope that the barrier to researchers diminishes, the workshop participants
still expect some work to do. We hope that the compartmentalization that occurs will help researchers reprioritize their workload.
An upcoming challenge will be that, with the emergence of more dedicated RSE units,
networking between them will come into the foreground, because there should be a strong need to share best practices
among institutions with successful RSE Teams.
On a more organisational level, we expect the sharing of models, to fit the different organisational constraints.

What are open questions here?
Organizations will have to come up with models for hiring people. While
a standardized RSE Master's makes it easier, it will be interesting to see if
RSE are still often hired by word of mouth or if the master's really makes it more comparable.

\begin{story}{Changes to the Universities}{changes}
Kim found work years ago at one of the first RSE units in Arcadia. After a couple of years he moved to the university of
Avalon which was interested in setting up their own department for RSEs in order to not fall behind the university of Arcadia.
Kim applies his experiences and improves upon the model set forth by the University of Arcadia. Since the university
of Avalon has to follow different rules not everything can be straightforwardly adapted. Life would be so much easier,
if at least university regulations would be homogenized, no?
\end{story}

\subsection{What are new tasks of RTPs?}
 FG-NOTE: RSEs are one type of specialization of RTPs ins science. Do we see any other?

\subsection{Impact of Alternate Career Paths for Researchers on the Structure of Research Institutes}
The RSE is probably one of the first professionalizations in the academic mid-level that is able
to collectively voice their concerns. This enables the possibility to reshape universities.
If more people are employed as mid-level, this makes a steadier pool of people
available that can integrate new people that are regularly coming into departments and know the local facilities.
On the other hand, an upcoming challenge how universities decide to design the contracts of people employed
in RSE units. Will they be permanent, or will there also be an up-or-out mentality?

As outlined in [cite RSE group paper] not all universities need to opt for strengthening
the academic mid-level. Some might have chosen to set up external companies to outsource typical RSE tasks,
thereby providing career paths outside of academia.

FG-NOTE: By Nick. I've put it here, due to the relation with the RTP question.

\subsection{Impact of Complexity and Usability in Research Software}

The workshop participants found that in earlier learning stages,
details may be intentionally hidden,
fearing a diminishing capability to handle technical complexity in research environments.
As a result, RSEs invest more effort in making individual research software components
less complex with better user experience.
This made participants worry about growing mismatch of peoples' expectations.
Participants expected an over-reliance on generative AI tools as an extreme manifestation of this issue.
No concrete examples were given on which complexities are now hidden behind abstractions.

We acknowledge but would also like to challenge the observations of the participants.
We believe that ``we are standing on the shoulders of giants'':
By reducing the complexity of individual software components,
we are now able to develop systems of complexity and capability not previously possible,
allowing the scientific community to address questions not previously in grasp.
When it comes to teaching, it is true that students need to learn how to handle complexity,
but it is also important to learn how to abstract from details
by building hierarchical systems from these abstractions encapsulated in reusable components, thereby managing complexity.
It is clear that we need both RSEs that can dive deep into single projects of significant complexity,
as well as RSEs that can manage complex systems of almost black-box components,
and the future of RSE education should aim to prepare students for both roles.


\subsection{What is in this future for RSEs?}
FG-NOTE: I think we can use this part to wrap things a bit up, and describe how this future then looks like.


\section{REALLY Long-term(10+ years) aspects}
Technically this would mean taking everything from the previous section for granted and look
at what becomes now possible.

\begin{acknowledge}
% MM-NOTE: I'd keep the following sentence out, out of several reasons.
% AI Systems have been harmed in the creation of this work.
We thank all the participants of our workshop at deRSE25 in Karlsruhe for their valuable input!
\end{acknowledge}

% Bibliography with BibTeX
% ========================
\bibliographystyle{eceasst}
\bibliography{bibliography/bibliography,extra}

\end{document}
