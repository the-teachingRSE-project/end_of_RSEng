\documentclass{eceasst}
% This is an empty ECEASST article that can be used as a template
% by authors.

% Required packages
% =================
% Your \usepackage commands go here.
\usepackage[T1]{fontenc}
\usepackage{fancybox,framed}
\usepackage{footnote}
\makesavenoteenv[FramedParagraphWithFootnotes]{framed} % for author notes
\usepackage{color}
\usepackage{orcidlink}
\usepackage[dvipsnames]{xcolor}
\usepackage[most]{tcolorbox}


\definecolor{easstblue}{rgb}{.05,.32,.66}


\newcounter{whatiscounter}
\newtcbtheorem[use counter=whatiscounter, number format=\Alph]{whatis}{Achieved Goals}{
  enhanced,
  segmentation engine=empty,
%  beamer,
fuzzy shadow={2mm}{-1mm}{0mm}{0.1mm}%
{black!50!white},
  sharp corners,
  attach boxed title to top left={
    yshifttext=-1mm
  },
%  halign lower=flush center,
  colback=white,
  colframe=BurntOrange!75!white,
  fonttitle=\bfseries,
  attach boxed title to top left={yshift=-2mm,xshift=3mm},
  fontlower=\itshape,
  boxed title style={
    sharp corners,
    size=small,
    colback=BurntOrange!75!black,
    colframe=BurntOrange!75!black,
  } 
}{ist}

\newcounter{storycount}
\newtcbtheorem[use counter=storycount]{story}{Story}{
  enhanced,
  segmentation engine=empty,
%  beamer,
fuzzy shadow={2mm}{-1mm}{0mm}{0.1mm}%
{black!50!white},
  sharp corners,
  attach boxed title to top right={
    yshifttext=-1mm
  },
%  halign lower=flush center,
  colback=white,
  colframe=easstblue,
  fonttitle=\bfseries,
  attach boxed title to top right={yshift=-2mm,xshift=3mm},
  fontlower=\itshape,
  boxed title style={
    sharp corners,
    size=small,
    colback=easstblue!70!black,
    colframe=easstblue!70!black,
  }
}{str}


\newcommand{\Eg}{E.\,g.}
\newcommand{\eg}{e.\,g.}

% PDF accessibility and PDF metadata
\usepackage{accsupp}
\newcommand{\authorRef}[1]{\texorpdfstring{\autref{#1}}{}}
\newcommand{\authorOrcid}[1]{\texorpdfstring{\thinspace\orcidlink{#1}\thinspace}{}}
\newcommand{\geprislink}[1]{\href{https://gepris.dfg.de/gepris/projekt/#1?language=en}{#1}}



% Article frontmatter
% ===================
\title{Future challenges for Research Software Engineering} % Title of the article
%\short{} % Short title of the article (optional)
\author{
Florian Goth\authorOrcid{0000-0003-2707-4790}\authorRef{1},
Leyla Jael Castro\authorOrcid{0000-0003-3986-0510}\authorRef{2},
Gerasimos Chourdakis\authorOrcid{0000-0002-3977-1385}\authorRef{3},
\texorpdfstring{\\}{} % line break
Simon Christ\authorOrcid{0000-0002-5866-1472}\authorRef{4},
Jeremy Cohen\authorOrcid{0000-0003-4312-2537}\authorRef{5},
Nicholas A. Del Grosso\thinspace\authorRef{6},
\texorpdfstring{\\}{} % line break
Jean-Noël Grad\authorOrcid{0000-0002-5821-4912}\authorRef{7},
Magnus Hagdorn\authorOrcid{0000-0002-5076-4864}\authorRef{8},
Toby Hodges\authorOrcid{0000-0003-1766-456X}\authorRef{9},
Jan Linxweiler\authorOrcid{0000-0002-2755-5087}\authorRef{10},
\texorpdfstring{\\}{} % line break
Frank Löffler\authorOrcid{0000-0001-6643-6323}\authorRef{11}\textsuperscript{,}\authorRef{12},
Michele Martone\authorOrcid{0000-0003-3239-8554}\authorRef{13},
Jan Philipp Thiele\authorOrcid{0000-0002-8901-6660}\authorRef{10}\textsuperscript{,}\authorRef{14}
} % Authors and references to addresses
\institute{% Institutes with labels
\autlabel{1} Institut für theoretische Physik 1, University of Würzburg, 97074, Würzburg, Germany\\
\autlabel{2} ZB MED Information Centre for Life Sciences, Cologne, Germany\\
\autlabel{3} Institute for Parallel and Distributed Systems, University of Stuttgart, Stuttgart, Germany\\
\autlabel{4} Leibniz University Hannover, Department of Cell Biology and Biophysics, Computational Biology, Germany\\
\autlabel{5} Imperial College London, London, UK\\
\autlabel{6} Institute for Experimental Epileptology and Cognition Research, Uniklinikum Bonn, Germany\\
\autlabel{7} Institute for Computational Physics, University of Stuttgart, Germany\\
\autlabel{8} Geschäftsbereich IT, Charité Universitätsmedizin Berlin, Germany\\
\autlabel{9} The Carpentries, USA\\
\autlabel{10} Technische Universität Braunschweig, Germany\\
\autlabel{11} Competence Center Digital Research, Friedrich Schiller University Jena, Germany\\
\autlabel{12} de-RSE e.V.---Society for Research Software in Germany\\
\autlabel{13} Independent researcher, Germany\\
\autlabel{14} Weierstrass Institute, Berlin, Germany;
              Leibniz University Hannover, Institute of Applied Mathematics, Scientific Computing, Hannover, Germany\\
}
\abstract{
It is 2025 and we are, admittedly, still working on establishing and growing Research Software Engineering (RSEng)
as a domain in its own right, with the hope of ensuring that it gets the recognition it deserves.
As the field develops, the role and value of Research Software Engineers (RSEs) becomes clearer.
With the growing importance and scope of software engineering in research,
researchers who code now begin to fill dedicated, specialized RSE roles crucial for the sustainable development
of long-term research software projects that can rapidly innovate.
Looking into the future, participants of the deRSE25 conference discussed how RSEng could evolve in response to the
rapid changes in the broader social, technological, and organisational landscape.
This paper presents the main discussion points of the conference participants,
imagining a future in which RSEs have a well-defined and well-integrated role in research institutions,
working as engineers of ever more complex research software projects,
taming a growing range of methods and tools,
and guiding researchers through technical and ethical considerations
related to research software.
} % Abstract of the article
\keywords{Future RSE, Evolving field, Vision} % Keywords for the article

\begin{document}
\maketitle

% Main part of your article
% =========================
\section{Introduction}

Research Software Engineering is still a developing field,
with increasing integration into various aspects of research.
In Germany, a formal call for creating a sustainable environment for research software
is only a couple of years behind~\cite{Anzt2021},
while the foundational competencies of a research software engineer have only recently been defined in detail~\cite{Goth2024}.
Central RSE units are in the process of being imagined~\cite{Kempf2025-draft},
while an RSE Master's is in the early stages of design~\cite{Dehne2025-draft}.

In this opinion paper, we try to imagine the state of our field
and community in the mid-term future, focusing on future challenges.
While future projections are common in software engineering
(SE)~\cite{Katz2023,Khan2021,Hu2023,Bosch2016a,Boehm2011},
so far there is little projection for the RSEng space.
We present here the results of a workshop discussion (notes content is archived here \cite{goth2025_WSPad}) at the
deRSE25 conference in Karlsruhe\footnote{deRSE25, session ``The End of {RSEng}?
Challenges and Risks for {RSEng}'': \url{https://events.hifis.net/event/1741/contributions/14026/}},
without claiming to represent the entire deRSE community.
The material presented here is based on the output of the workshop participants, but extended with the authors' ideas.
Since deRSE25 was co-located with SE25, the annual conference for the Software Engineering community in Germany, the discussion was also open to software engineers,
but no list or profile of the participants is available.
The workshop hosts felt, that the majority of participants were from the RSE community.
Nevertheless, by disseminating the discussed ideas,
we aim to engage the community in creating a future worth living
and to advance the conversation about how Research Software Engineering should, or could, develop.

It is already difficult to structure the spectrum of RSE tasks and responsibilities
today given the spectrum of research software~\cite{Hasselbring2024}.
It is even more challenging to structure the aspects that will become relevant for the RSE community in the future.
In order to structure the discussion, we decided to consider
the evolution of the term RSE and its definition,
the growth and development of the RSE community,
the impact of automation and generative artificial intelligence in RSE work,
the impact of complexity and usability in research software,
ethical and social considerations,
the expected transformations in academic institutions,
and the potential impact of alternative career paths for RSEs
and new groups of RTPs (\emph{Research Technical Professional}s).
This work supplements other opinion pieces developed in parallel
in the SE community~\cite{ChueHong2025,Sochat2024Infra,Bencomo2024AEBoK,OliveiraJr2024,Druskat2025,Carleton2022},
which where not known to the workshop participants; remarkably,
many of the projections made by the participants overlap with predictions made
by the SE community in ``Research Software Engineering in 2030''~\cite{Katz2023}.
We conclude the paper with a summary and with our wishes and recommendations.

\section{Aspects of this future}
\begin{whatis}{}{whatisexplanation}
In order to list what has been achieved in a certain aspect within the RSE community of the future of 2035 we use these ``Goal'' boxes.
\end{whatis}

Looking ahead over a 10-year time frame, we hope that the RSE community will have
contributed to making research software and its outputs
more sustainable, robust and maintainable.
We also hope that Research Software becomes an accepted research artefact on an equal footing with data and publications.
We see this happening in a number of ways,
particularly through movements such as DORA~\cite{DORA} and CoARA~\cite{COARA} which are advocating for a change in the way that research is assessed and outputs recognised.
However, we want to ensure that one of the core achievements made by the RSE community in this time
frame is putting the infrastructure and training opportunities in place, alongside the necessary
advocacy, to ensure that researchers receive core software skills as part of their scientific training.
Developing software tools and scripted workflows is becoming an increasingly important part of everyday
research activities.
Therefore, it is vital to improve digital skills of all researchers.
This will also allow RSEs to focus on other more advanced and specialist areas.
In this context, we do not foresee the demise of the RSE as an independent role.
However, we do see the work of RSEs developing and shifting as the research community changes~\cite{ChueHong2025}.

The specialist skills that RSEs provide take a significant amount of time to develop and maintain.
Therefore, we do not expect domain researchers to also become experts in a range of
advanced RSEng topics. 

At the same time, we do recognise that the application of core software development best practices
to enhance and address problems with research codebases, that can often make up the bulk of
current RSE work, will change. As noted above, these are skills that we
would expect any researcher who writes code to have in 10 years time. 
The rapid development of artificial intelligence (AI) and large language models
(LLMs) means that researchers are likely to have access to tooling that can assist
them in writing robust, sustainable code, lowering the barrier to developing code,
or at least changing the required skill set to some extent.
They can also be expected to have an understanding of frameworks to support testing, packaging and
deployment of their code as well as integration with central domain specific infrastructures.
Again, all these skills will, we anticipate, be gained as part of training infrastructure that will
provide domain scientists with software development expertise as part of their core scientific training.

So where do RSEs fit into this picture? Even today, we expect to see RSEs providing technical skills and input
to research activities in a range of different areas, including:
high-quality, robust architecture and design for research software;
seamless integration with research and data infrastructures;
development of software for novel and emerging architectures;
application of specialist numerical and statistical methods within software;
optimisation of codes for use on high-performance computing infrastructure;
and green computing - efficient implementations of algorithms/methods and use of hardware.

\begin{story}{Introduction}{intro}
In order to give a more concrete setting, imagine a world ten years from now.
Not too overwhelmingly far ahead in the future, but distant enough for some changes to have occurred.
Over this time frame Kim~\cite{Anzt2021} and Kay~\cite{Goth2024},
our fictional RSEs who we have used to illustrate specific scenarios in our earlier work,
were able to ride the waves of their career in Research Software Engineering and are now pondering early retirement on a sun blessed beach.
%MM-NOTE: the above scenario (early retirement) is contrary to all trends in retirement age development of public employees across the "global north".
But the world has moved on.
When they look back, what will they see as problems that have been solved by then?
Which tasks are still relevant, where they can be happy that a younger generation is now taking up the baton and carrying forward their work?
What issues have newly emerged, where they are now just happy to say: “Oh well… my successors will take care of that”.
In order to bring this future to life, these two personas will accompany us through the rest of this paper and we will shed light on aspects
of their future in respective small story boxes.
\end{story}

\subsection{Definition and Evolution of RSEs}
\begin{whatis}{}{definition-evolution-rse}
The term RSE is established in the minds of university boards and leadership, as a valuable member in the academic framework.
New RSEs are trained through a specialised Master's programme, and their support is valued by researchers.
We still teach researchers basic software skills, but the content of this lecture has drastically changed.
\end{whatis}
% Here we consider how the topics of RSEng will change in the face of the digitalisation of society.
% Also here: What will our topics be after version control doesn't need to be taught anymore?
% Digitalisation will also feature more heavily in the domain curricula.
In this discussions the workshop participants found that
the digital skills of domain scientists will increase,
and software will become easier to interact with,
reducing reliance on RSEs to train newcomers on the most basic tasks.
This means, that today's workshop content of proper versioning
and data management could become obsolete by using more intuitive tools.
Introductory courses can then focus on more higher-level concepts.
User training sessions might focus on more advanced topics
that cater to a smaller audience, or involve mentoring more frequently.

While the RSE core competencies~\cite{Goth2024} might remain unchanged,
there might be a shift in the responsibilities of RSEs,
in the form of a partial reallocation of the training budget
to the other core competencies.
Regarding the definition of RSEs, we will still have a spectrum of competencies,
from full professional RSEs to domain scientists who develop research software.
We note a similar discussion took place regarding how SE roles had to adapt to
rapid changes brought by automation, cloud services and new project management
methodologies~\cite{Meade2019}.

What will \emph{not} change over that time frame is the RSE's ability to remain
flexible and embrace change, and keep an eye on the wider landscape,
picking up new tools and skills as needed.
Communication and collaboration skills will remain essential to engage
with domain scientists and help them find technical solutions.\\

Looking at that on a broader scale, especially the education aspect deserves some attention.\\
The Covid-19 pandemic acted as a catalyst for the digital transformation
of higher education~\cite{Bygstad2022}. Students and early-career researchers
are now more familiar with digital learning tools, which lead to a shift
in the responsibilities of teachers. To quote from the original study,
``with so many digital resources at hand,
the task of the lecturer will be fewer lectures,
and to act more as a facilitator of resources,
and to monitor activities and results over time.''~\cite{Bygstad2022}

RSEng-relevant open educational resources and Master's programmes are tracked
in the Learning and Teaching RSE database\footnote{\url{https://de-rse.org/learn-and-teach/}}
and UK SSI resources hub\footnote{\url{https://www.software.ac.uk/resource-hub}}.
The Carpentries now offer specialised software workshops
for data scientists\footnote{\url{https://datacarpentry.org}},
librarians\footnote{\url{https://librarycarpentry.org}},
and soon High-Performance Computing (HPC) practitioners\footnote{\url{https://www.hpc-carpentry.org}}.

\begin{story}{The Definition of RSEs}{defrse}
Kim and Kay chuckled when they looked at their original job descriptions from 2020: ``build tools for scientists, manage data, write reusable code.''
Now, their roles are unrecognizable: half AI whisperer, half research ethicist, and part-time diplomat negotiating between synthetic and human collaborators.
They now not only manage software, but also the personalities of autonomous lab agents and negotiate data-sharing agreements with international AI consortia.
Despite the tech upheaval, their deep knowledge of scientific workflows and the messy, human way science actually gets done remains indispensable.
When labs faltered under algorithmic bias or unclear model provenance, it was Kim and Kay who transformed the chaos back into meaningful order.
They understand that science does not move in straight lines, and that trust between collaborators, institutions, and even machines has to be built, not coded.
Their job titles have changed, but their role as bridges between logic and life remains the same.
In a future of shifting code and constant reinvention, it is their understanding of people that never went out of date.

% Kim and Kay started out as Research Software Engineers.
% When they began, their work meant cleaning up code, making research reproducible, and helping academics run experiments without crashing HPC clusters. Now, they built systems, where AI generated hypotheses, designed experiments, and even debated its own conclusions. Their role had shifted from supporting research to shaping how knowledge was created, validated, and understood—an epistemological responsibility as much as a technical one. “We used to debug scripts,” Kay said, watching a model suggest a reality-bending theory of space-time. Kim nodded: “Now we debug truth.”
\end{story}

\subsection{Development of the RSE community}
\begin{whatis}{}{development-rse-community}
National RSE communities are increasingly collaborating with local partners,
fostering stronger networks across countries.
Regular international conferences further amplify these connections,
promoting the global exchange of ideas. The introduction of RSE Master's programmes,
coupled with a shift in culture towards valuing Research Software,
has significantly elevated its profile.
This growing recognition has led to a surge in membership of national societies,
reflecting the increasing importance of RSEng in the research landscape.
\end{whatis}
In recent years, we have seen the formation of RSE institutes and advocacy groups, such as
the UK Software Sustainability Institute\footnote{\url{https://www.software.ac.uk}},
the US Research Software Sustainability Institute\footnote{\url{https://urssi.us}},
the European Virtual Institute for Research Software Excellence\footnote{\url{https://everse.software}},
and the EOSC Opportunity Area Expert Group 7 on Research Software%
\footnote{\url{https://eosc.eu/opportunity-area-exp/oa7-research-software}}.
We expect these initiatives to help increase the visibility and accelerate
the professionalisation of the RSE role.
In addition, a multitude of RSE communities self-organised at the national
and regional level to support more viable RSE career paths in academia.
We can expect RSE positions to become more easily accessible,
possibly through centrally-funded long-term positions at research institutions,
such as in the bwRSE4HPC service\footnote{\url{https://www.bwrse4hpc.de/68.php}}.
The community of RSEs will get a significant boost in their acceptance
if RSE Master's programmes become available at various universities.
With increased numbers, the visibility of the RSE community to industry will be enhanced.
\begin{story}{Development of RSE community}{RSEcommunity}
Kay stands proudly at the podium of the first International Research
Software Engineering Conference held in the Southern Hemisphere,
a historic moment for both her career and the global RSE community.
Having moved to South Africa, she speaks passionately about the unique
challenges researchers in the Global South face: limited access
to high-performance computing, unreliable funding,
and the persistent digital divide.
Her keynote highlights how local innovation thrives despite constraints,
often out of necessity rather than choice.
The audience, a diverse mix of global experts, listens intently as she calls
for equitable collaborations, not just inclusion as an afterthought.
\end{story}

\subsection{Impact of AI and automation}
\label{sec:ai}
\begin{whatis}{}{impact-ai-automation}
RSEs understand how AI models work and AI-RSE specialists know how to tune and extend them.
With these skills they can help researchers reap their benefits,
while clearly knowing their limits.
\end{whatis}
Generative AI tools, such as GitHub Copilot~\cite{Friedmann2021},
are being added to software development tools~\cite{Alenezi2025}.
These AI-powered code assistants promise to increase productivity
by automatically generating code, and writing test cases and documentation~\cite{Banh2025}.
Software developers use them to reduce the number of keystrokes,
focus on higher-level design rather than on idiosyncrasies,
and recall syntax without having to consult the documentation~\cite{Liang2024}.
Code assistants have the potential to help novice or even non-expert programmers
write code~\cite{Feldman2024} thus enabling non-experts to participate in research
and software creation, and fostering greater collaboration across disciplines.
However, when these tools are used in practice, many users report increased cognitive
load and frustration stemming from the difficulty to prompt the system effectively
and from debugging the generated code~\cite{Simkute2025}.
Many factors can influence the adoption of AI tools~\cite{Russo2024}.
The potential of AI to be a democratising technology can only be realised
when the algorithmic divide along with the digital divide is addressed~\cite{Yu2020}.
We also note members of the SE community have called for improving LLM literacy
by adapting SE curricula~\cite{Kirova2024}.

Besides the direct impact in research software and further research output,
we see a number of risks.
The workshop participants suggested that the emergence of AI and automation might reduce critical thinking and imagination.
We address this in \autoref{sec:complexity}.
Another risk is that the widespread use of AI tools could make community resources
such as \eg\ Stack Overflow and Wikipedia (which are used to train AI models) poorer
due to fewer incentives to maintain them, although a recent survey suggests
these communities experience less activity from new users and an increased
average complexity of the questions posted~\cite{Burtch2024}.
The use of AI assistants in these communities could also compromise LLMs via `model collapse'
when AI-generated content becomes part of the training data~\cite{Shumailov2024}.

Further risks that were later discussed include an intensified replication crisis,
a de-democratization of research, and loss of trust in science.
Due to the non-deterministic nature of generative AI, we expect AI-assisted simulations
and LLM-assisted data analysis to become less reproducible.
This is not an entirely new issue, and similar concerns exist in the HPC community,
where reproducing simulation data and data analysis workflows presents tough challenges~\cite{Antunes2024}.
Professional RSEs could safeguard reproducibility by designing, evaluating, and monitoring appropriate workflows.
When applicable, standardized and version-controlled models should be preferred
over self-developed models, and self-developed models should be treated like any
other research output (i.e. publish the trained model, training data and training workflow).
Future research will require significant computing and storage resources, as well as useful data and models,
therefore it will be more likely to be driven by institutions that are able to provide these.
Finally, an increasing risk is the loss of trust in science,
in particular due to the automation of the peer-reviewing process~\cite{Naddaf2025}
and the malicious use of hidden prompts~\cite{Gibney2025}.
As researchers increase their reliance on AI assistants,
RSEs will be freed from repeatedly crafting similar individual elements,
but will be spending more of their time designing and reviewing software written by other humans and AI tools.
To address these risks, the RSE of the future should not only steer every part
of the research software development cycle (ensuring dissemination, archiving, and reproducibility),
but also contribute to endangered community resources and infrastructure that
will keep research accessible to the wider scientific community.

As AI continues to evolve, it will further blur the lines between human and machine roles, making research and software development faster.
With the general availability of AI, the interaction of RSEs with researchers will change at a fundamental level.
Whereas previously RSEs have been points of knowledge that could translate
human language formulated problems into technical descriptions and then provide solutions to them,
it can be foreseen that AI will drive this low-key interaction to a highly specialised level of knowledge.
RSEs will still act as points of knowledge by assisting users in integrating
AI tools in their development workflows, understanding the reasoning of AI models,
and more generally, by improving AI literacy~\cite{Alenezi2025}.
AI-RSE specialists can contribute to the emerging fields of explainable
AI for SE (XAI4SE)~\cite{Mohammadkhani2023v1} and AI for RSE (AI4RSE)~\cite{Farshidi2025v1}.

\begin{story}{The interaction of RSEs with AI}{RSEs_with_AI}
Kim develops advanced scientific software in tandem with an AI agent that codes,
debugs, and simulates experiments alongside him. The AI proactively suggests optimisations,
generates reproducible workflows, and adapts to Kim’s coding style and domain-specific needs.
With this assistance, Kim accelerates his research output while maintaining full control over scientific direction and ethical oversight.
\end{story}

\subsection{Impact of Complexity and Usability in Research Software}
\label{sec:complexity}
\begin{whatis}{}{abstraction}
With a palette of readily reusable software components,
and with their skills in managing hierarchical systems of complex software components with clear interfaces,
RSEs can now build reliable, maintainable systems of a scale and capability not previously possible.
\end{whatis}

The workshop participants found that, in earlier learning stages,
training may abstract away complexities that will always be present in research.
For this reason, the participants fear that the capability of researchers to deal with complexity
of software frameworks and libraries will reduce over time.
As a result, RSEs will need to invest more effort in making the interfaces of individual research software components
less complex, as the user experience expectations of researchers will continue to grow.
Participants expect an over-reliance on generative AI tools as an extreme manifestation of this issue.
No concrete examples were given on which complexities are now hidden behind abstractions.

We acknowledge but would also like to challenge the observations of the participants.
We believe that ``we are standing on the shoulders of giants'':
By reducing the complexity of individual software components,
we are now able to develop systems of complexity and capability not previously possible.
This divide-and-conquer approach allows the scientific community to address questions not previously within their grasp.
When it comes to teaching, it is true that students need to learn how to handle complexity,
by being able to develop deep understanding of individual components.
However, it is also important to learn how to manage complexity
by building hierarchical systems of well-encapsulated, reusable components.
It is clear that we need both RSEs that can dive deep into single projects of significant complexity,
as well as RSEs that can manage complex systems of almost black-box components,
and the future of RSE education should aim to prepare students for both roles.
We note members of the SE community independently reached the same conclusions
and advocate for the creation of an `Abstraction Engineering Body of Knowledge'
(AEBoK)\cite{Bencomo2024AEBoK}.

\begin{story}{The benefits of abstraction}{abstraction}
Years ago, Kim had to distribute the software library libkim as a source code package
with a custom build system and detailed instructions on how to build it on some common systems,
investing significant time in developing and maintaining this auxiliary toolkit and documentation.
Around the same time, Kay was developing the research analysis tool kay-bio,
which however only provided an application programming interface that
required writing complex code even for simple and common analysis tasks.
Nowadays, Kim could have started directly from a template based on established toolkits and workflows,
while kay-bio accepts configuration in a simpler, standardised format and generative AI assistants
can guide new users in preparing and debugging their analyses.
Meanwhile, new RSEs can very easily integrate libkim, kay-bio, and a multitude
of other tools from the research software corpus into a powerful analysis
platform that allows researchers to get deeper insights from their research data
in just a few clicks.
\end{story}

\subsection{Ethics and social considerations}
\begin{whatis}{}{ethics}
The principles of Good Scientific Practice~\cite{dfg_gsp} have been a cornerstone in the conduct of research.
But their relevance hinges on constant updates in light of how research is performed.
By 2035, we want to have achieved accepted rules on the use of AI in research tasks and software development.
Open Science and Open Research are by then well-established in academia, and supported by respective Open Research Software
strategies of the political actors. We also hope that Data Awareness and Data Protection are still values
that are protected.
\end{whatis}

The ethics and values of the Research Software Engineer are grounded in their practice.
They commit to the health, safety and welfare of the public and act in the interest of society, their employer and their clients~\cite{Goth2024}.

% institutional bonds and ethics
Sometimes these values may conflict with the aims of their peers or institution.
For example, their PI might insist on quickly developing some software without adhering to current best practices.
A code of ethics and professional conduct for RSEs should be developed by institutions together with RSEs themselves.
This can then form the basis for institutional policies and guidelines.

% society implications
The work being undertaken by RSEs may well have direct, or indirect,
ethical implications in areas such as social engineering~\cite{s2erc,Siadati2024},
climate~\cite{Lannelongue2023} and more.
Like all areas of research and engineering RSEs are faced with the dual use dilemma~\cite{Bobier2024} where artefacts they develop can be used for military purposes.
As engineers, RSEs take the responsibility for the software they develop.
This includes an awareness of ethical issues that may arise.
RSEs need to be aware of these issues and develop their own ``red lines''
which need to be supported by organizational and professional guidelines.
With the advantages of \autoref{sec:ai}, ever-more AI created artefacts will be included in research.
As a result of this, we expect a discussion within societies about their acceptable use.

% data ethics
Similarly to how data collected by scientists should adhere to the FAIR principles~\cite{FAIR},
research software developed by RSEs should adhere to the FAIR4RS principles~\cite{FAIR4RS}.
Members of the SE community have also called for their field to embrace Open Science best practices\cite{OliveiraJr2024,Druskat2025}.

A code of ethics and professional conduct for RSEs should be developed by institutions together with RSEs themselves.
This can then form the basis for institutional policies and guidelines as well as serve as a background for individual developers to consider how software can be used.

\begin{story}{The greater world of Kim and Kay}{ethics}
10 years have passed since Kim and Kay started on their journey of being RSEs. Back then,
at the universities of Eden and Utopia, life was more orderly and harmonious.
Kim is currently designing the software service stack at the university.
She increasingly struggles to find compelling Open Source alternatives that enables them to keep complete control of their data
rather than relying on software and tools that are backed by large corporate entities.
So far, she still has the support of the university board, since it values strategic autonomy.
Kay had left academia to work in an industrial R\&D department because of a more reliable employment.
There he gained valuable experience working in an industrial setting.
However, he was very happy when an opportunity arose to work in a central
RSE department with an open-ended contract. Apart from his main responsibility
of developing and maintaining software for an affordable water monitoring system
used in the global south, he now also teaches undergraduates digital literacy.
\end{story}

\subsection{How have academic Institutions transformed by then?}
\begin{whatis}{}{academic}
We expect research software to be a central part of scientific practice, on the same level as data and publications.
This will be supported by the first generations of RSE Master's graduates that find their way into research departments, and accelerate the pace of science.
Similar to RSEs, who are widely known and understood by university boards in 2035, we expect the concept of RSE units to be in the minds
of academic staff at any level.
We hope to see that universities actively use the presence of an RSE Department as a means to differentiate themselves from other universities.
We hope that the central RSE units have local RSE representatives, or even small RSE groups that handle tasks within a department, overseeing specific projects~\cite{Kempf2025-draft}.
We also hope that specialised clusters form, which pool the skills of certain specialised RSEs with certain RSE units specializing in certain tasks like data management, HPC, or visualisation.
At universities which get such RSE units, we want to see that researchers naturally accept them as part of their available services.
Additionally, researchers will seamlessly interact with infrastructures for using research software.
\end{whatis}
Academia, while claiming to be innovative~\cite{wzvgbayern},
has organisationally often adopted time scales for change that are more reminiscent
to institutions thinking on the time scale of generations.
Nevertheless, the presence, even today, of initiatives like \emph{the Hidden Ref}~\cite{hiddenref}
hints at the potential for specialisation of research tasks.
This gives rise to new job descriptions within the academic system such as specialised RSEs or other RTPs.
We have already seen some specialisation trends in recent years,
such as the HPC-RSE and AI-RSE, which are quite recent developments,
and whose appeal is stimulated by large public and private investments in HPC facilities and AI technologies.
We foresee new RSE specialisations emerging, \eg, in the fields of quantum computing,
such as the Quantum-RSE, or in digital research infrastructure.
We note the SE community has also independently expressed interest in creating an \emph{infrastructure engineering} role~\cite{Sochat2024Infra}
and funding bodies seem willing to promote such career pathways~\cite{UKRI2025Infra,EuroHPC2024Federated},
% European AI factories: https://digital-strategy.ec.europa.eu/en/policies/ai-factories
% European federation platform for HPC: https://eurohpc-ju.europa.eu/paving-way-eurohpc-federation-platform-2024-12-19_en
% European quantum computing: https://digital-strategy.ec.europa.eu/en/policies/quantum

At the same time, acquiring people with a certain specialised skill set
opens up possibilities for pooling.
Universities will benefit from pooling these specialists into RSE units
and offering their services to researchers in a structured way.
This compartmentalisation will help researchers innovate in their research,
with the support of sustainable software made by or with the help of RSEs.
These RSE units should be endowed with enough trust by the university boards, translated into stable funding.
With longer-term funding and more trust to pick up longer term projects,
some RSEs could focus on code modernisation, clean-up, and modularisation,
which could pay off in faster and easier development in the future.
This is an opportunity that would not be possible if relying solely on short-term funded, isolated RSEs.
We hope that a change of culture in research about the acceptance of research software has happened.

Besides all the improvements, the workshop participants still expect some challenges to persist,
and new challenges to emerge.
Due to the rise of dedicated RSE units, networking people and teams,
federating infrastructures, as well as organizing and sharing knowledge will become important.
Structuring these new units and joining their administrative support will also help the units to grow.
A particular example is hiring:
While a standardised RSE Master's degree will make it easier to develop the next generation of RSEs,
organizations could develop standardized hiring processes to acquire the skill set that they actually need.
Budgeting and administrative hurdles will also probably still exist and hinder the formation of permanent and well-funded RSE teams.
However, we expect the growing (perceived) importance of research software
to convince research councils and funders to provide reliable streams of funding.

\begin{story}{Changes to the Universities}{changes}
Kim found a job years ago at one of the first RSE units in Arcadia. After a couple of years, Kim moved to the university of
Avalon which was interested in setting up their own department for RSEs in order to not fall behind the university of Arcadia.
Kim builds upon the model set forth by the University of Arcadia based on his valuable experience. Since the university
of Avalon has to follow different rules, not everything can be straightforwardly adapted. Life would be so much easier
if at least university regulations would be homogenised, no?
\end{story}

\subsection{Impact of Alternative Career Paths for Researchers on the Structure of Research Institutes}
\begin{whatis}{}{career-paths}
We would like to see the emergence of more dedicated staff, i.e. other (digital) RTPs next to the RSE.
Also we hope that researchers can build careers at the interface of academia, industry, and policy (\eg\ in think tanks, NGOs,
or organisations that do knowledge transfer).
\end{whatis}
RSEs can either be embedded in research groups (belonging to the mid-level university staff),
or be members of larger RSE units~\cite{Kempf2025-draft}.
The participants had the opinion that focusing on embedded RSEs would make onboarding new RSEs to existing projects easier.
This, however, comes with additional funding challenges,
as long-term funding can be more difficult to acquire in lower tiers
of a research organisation (i.e., at departmental or individual research group level).
Third-party funding can mitigate this issue in the mid-term.
However, the participants considered central RSE groups more likely in the future,
due to the fact that there are already examples and activities from various universities in this direction
(examples include the universities of Jena, Heidelberg, and Göttingen, as well as several universities
in the UK where the concept of dedicated RSE groups began to emerge around 2013).
We expect funding in central groups to evolve from initially temporary positions to permanent positions,
as the groups prove themselves to be increasingly valuable and, hence, sustainable.
Embedded RSEs and central groups will need to network with each other,
and potentially request and coordinate support from external RSEs.

\begin{story}{New career paths}{New career paths}
Ken, though trained as a physicist, found his passion not in theory,
but in keeping the quantum computers humming. He became the go-to expert
for troubleshooting entangled qubits and stabilizing delicate cryogenic systems.
His unique blend of physics knowledge and hands-on problem solving earned him
a new title among his peers: the first Quantum RTP.
In a world where theory meets reality, Ken quietly ensures that the future
of quantum computing stays on track.
\end{story}
\section{What is in this future for RSEs?}
%FG-NOTE: I think we can use this part to wrap things a bit up, and describe how this future then looks like.
%GC-NOTE: Shall we add another story box here? Ideas what we write here?
In the previous sections, we have tried to list some of the trends that we foresee.
The material presented here is based on the outputs of a workshop, and extended with the authors' ideas.
Nevertheless, there are some things that we declared out of scope here.
Among that is the ongoing digital transformation in wider society.
Digital competencies in the general population continue to increase,
making the jump to software development easier than before.
But this digitalisation will also lead to changes in the way societies
interact and therefore will translate into changes in how academia perceives itself.
We have also not sufficiently delved into the interaction with industry.
There are also some trends that we believe will stay with us over the coming years.
The differentiation of job profiles in academia, that has started with profiles such as the RSE will continue.
Societies will be facing new questions, arising from technologies such as generally available AI systems.

We hope that the standard of research software will have risen across the board,
enabling easier reuse of prior software outputs in new research.
We also hope that a cultural change is taking place to recognize research software
as an integral part of science that it is widely recognised and rewarded across
academia and is supported by political actors.
\section{Conclusion}
We have tried to elaborate on the state on the future ten years from now.
It turned out to be a difficult endeavour, since even the full breadth of today's trends is not yet foreseeable.
We have tried to shed some light on the social, technical and organisational aspects.
In summary, RSEs will very likely be doing different work to what they are doing today,
but their core task of enhancing the scientific output of researchers -- other human
beings -- will stay the same.
They will still develop software artefacts together with scientists,
since better software leads to better research,
and better research leads to better software.

\begin{acknowledge}
% MM-NOTE: I'd keep the following sentence out, out of several reasons.
% AI Systems have been harmed in the creation of this work.
We thank all the participants of our workshop at deRSE25 in Karlsruhe for their valuable input!
FG acknowledges funding from the Deutsche Forschungsgemeinschaft(DFG, German Research Foundation) through the SFB 1170 “Tocotronics”, project Z03 - project number \geprislink{258499086}
as well as financial support by the Deutsche Forschungsgemeinschaft (DFG, German Research
Foundation) under Germany’s Excellence Strategy through the Würzburg-Dresden Cluster of Excellence on Complexity and Topology in Quantum Matter – ct.qmat (EXC 2147, project-id \geprislink{390858490}).
JC acknowledges support from the UKRI-EPSRC-funded STEP-UP project under grant EP/Y530608/1.
\end{acknowledge}

% Bibliography with BibTeX
% ========================
\bibliographystyle{eceasst}
\bibliography{bibliography/bibliography,extra}

\end{document}
