\documentclass{eceasst}
% This is an empty ECEASST article that can be used as a template
% by authors.

% Required packages
% =================
% Your \usepackage commands go here.
\usepackage{fancybox,framed}
\usepackage{footnote}
\makesavenoteenv[FramedParagraphWithFootnotes]{framed} % for author notes

% Article frontmatter
% ===================
\title{Future challenges for Research Software Engineering} % Title of the article
%\short{} % Short title of the article (optional)
\author{
Florian Goth\texorpdfstring{\autref{1}}{},
Leyla Jael Castro\texorpdfstring{\autref{1}}{},
Gerasimos Chourdakis\texorpdfstring{\autref{1}}{},
Simon Christ\texorpdfstring{\autref{1}}{},
Jeremy Cohen\texorpdfstring{\autref{1}}{},
Jean-Noël Grad\texorpdfstring{\autref{1}}{},
Magnus Hagdorn\texorpdfstring{\autref{1}}{},
Toby Hodges\texorpdfstring{\autref{1}}{},
Jan Linxweiler\texorpdfstring{\autref{1}}{},
Frank Löffler\texorpdfstring{\autref{1}}{},
Michele Martone\texorpdfstring{\autref{1}}{},
Jan Philipp Thiele\texorpdfstring{\autref{1}}{},
Nick del Grosso\texorpdfstring{\autref{1}}{}
} % Authors and references to addresses
\institute{\autlabel{1} Fantasy University} % Institutes with labels
\abstract{It's 2025 and we are, admittedly, still working on establishing and growing Research Software Engineering (RSE)
as a domain in its own right, with the hope of ensuring that it gets the recognition it deserves.
Nonetheless, as the use of digital tools becomes ever more pervasive in a researchers day-to-day tasks, there are already people suggesting that RSE will
cease to exist as a separate field, and the specialist skills that RSEs provide to support and undertake research will
become a core part of the tooling and skill set of researchers in almost all domains.
Are we already approaching "the end of Research Software Engineering"?! While we believe that it is certainly too early to
already herald the demise of RSE as a separate field,
we also recognise the need to acknowledge that the future is in constant flux and
it is worthwhile discussing how RSE will/could change in response to these upcoming challenges in an open discussion.} % Abstract of the article
\keywords{Keywords go here.} % Keywords for the article

\begin{document}
\maketitle

% Main part of your article
% =========================
\section{Introduction}
This opinion article stems from a Bof session at deRSE25 in Karlsruhe\cite{Goth2025EndRSEng}.
Therefore it stems from a discussion among the participants of this workshop and hence cannot claim that it represents the entire deRSE community.
Nevertheless we hope, that we succeeded in writing up something that
\begin{itemize}
\item you like to read,
\item gives you something to think about,
\item motivates you to work on creating this better future.
\end{itemize}
In order to structure the discussion, we decided on a imaginable setting ten years from now. 
In this world we will consider the dimensions of 
\begin{itemize}
\item Evolution of the term RSEs
\item The development of academic institutions
\item How will academic education have transformed by then
\item Ethics and social considerations
\end{itemize}
In each aspect we will quickly summarise on what we want to have achieved by then, and what will be the challenges that arise from that.



\section{The Situation}
In order to give a more explicit setting, imagine a world ten years from now.
Not too much, to be overwhelmingly far in the future, but distant enough for some changes to occur.
Over this timeframe Kim and Kay
were able to ride the waves of their career in Research Software Engineering and are now pondering early retirement on a sunset blessed beach.
But the world has moved on.
When they look back, what will they see as problems which are now solved?
Which tasks are persisting, where they are now happy that a younger generation is now taking up the baton and carries forward their work.
What issues have newly emerged, where they are now just happy to say: “Oh well… My successor takes care of that”.


\section{Aspects of this future}
It is difficult to structure the different aspects in describing the future. We decided to consider
\begin{itemize}
\item The evolution of the term RSE
\item How is education reshaped by the digital transformation
\item What will the impact of AI be to the work of RSEs?
\item Changes to the Academic Institutions
\item Ethics and social considerations
\item Impact of alternate career paths for RTPs
\end{itemize}

%\subsection{What do we want to have achieved by 2035}

\subsection{Definition and Evolution of RSEs}
\begin{framed}
How will the definition of an RSE evolve in ten years?

What will not change in the essence of an RSE ten years from now?
\end{framed}
% Here we consider how the topics of RSEng will change in the face of the digitalization of society.
% Also here: What will our topics be after version control doesn't need to be taught anymore?
% Digitalisation will also feature more heavily in the domain curricula.

The digital skills of domain scientists will increase,
and software will become easier to interact with,
reducing reliance on RSEs to train newcomers on the most basic tasks
(JN: for example, versioning and archiving?).
User training sessions might focus on more advanced topics
that cater to a smaller audience, or involve mentoring more frequently.

Regarding the definition of RSEs, we will still have a spectrum of competencies,
from full professional RSEs to domain scientists who develop research software.
While the RSE core competencies\cite{Goth2024} might remain unchanged,
there might be a shift in the responsibilities of RSEs,
in the form of a partial reallocation of the training budget
to the other core competencies.

What will \emph{not} change over that timeframe is the RSE's ability to remain
flexible and embrace change. Keep an eye open and pick up tools and skills
as needed as we go along.
Communication and collaboration skills will remain essential to engage
with domain scientists and help them find technical solutions.

\begin{FramedParagraphWithFootnotes}
JN-NOTE:
The Covid-19 pandemic acted as a catalyst for the digital transformation
of higher education\cite{Bygstad2022}. Students and early-career researchers
are now more familiar with digital learning tools, which lead to a shift
in the responsibilities of teachers. To quote from the original study,
``with so many digital resources at hand,
the task of the lecturer will be fewer lectures,
and to act more as a facilitator of resources,
and to monitor activities and results over time.''\cite{Bygstad2022}

RSE-relevant open educational resources and master's programmes are tracked
in the Learning and Teaching RSE database\footnote{\url{https://de-rse.org/learn-and-teach/}}
and UK SSI resources hub\footnote{\url{https://www.software.ac.uk/resource-hub}}.
The Carpentries now offer specialised software workshops
for data scientists\footnote{\url{https://datacarpentry.org}},
librarians\footnote{\url{https://librarycarpentry.org}},
and soon HPC practitioners\footnote{\url{https://www.hpc-carpentry.org}}.
\end{FramedParagraphWithFootnotes}

\begin{FramedParagraphWithFootnotes}
JN-NOTE:
In recent years, we saw the formation of RSE institutes and advocacy groups, such as
the UK Software Sustainability Institute\footnote{\url{https://www.software.ac.uk}},
the US Research Software Sustainability Institute\footnote{\url{https://urssi.us}},
the European Virtual Institute for Research Software Excellence\footnote{\url{https://everse.software}},
and the EOSC Opportunity Area Expert Group 7 on Research Software%
\footnote{\url{https://eosc.eu/opportunity-area-exp/oa7-research-software}}.
We expect these initiatives to help increase the visibility and accelerate
the professionalisation of the RSE role.
In addition, a multitude of RSE communities self-organised at the national
and regional level to support more viable RSE career paths in academia.
We can expect RSEs to become more easily accessible,
possibly through centrally-funded long-term positions at research institutions.
\end{FramedParagraphWithFootnotes}

\begin{FramedParagraphWithFootnotes}
JN-NOTE:
New RSE specialisations might arise.
HPC-RSE and AI-RSE are quite recent developments,
whose appeal is stimulated by large public and private
investments in HPC facilities and AI technologies.
We could imagine new RSE specialisations to emerge in the fields
of quantum computing and federated online platforms.
% European AI factories: https://digital-strategy.ec.europa.eu/en/policies/ai-factories
% European federation platform for HPC: https://eurohpc-ju.europa.eu/paving-way-eurohpc-federation-platform-2024-12-19_en
% European quantum computing: https://digital-strategy.ec.europa.eu/en/policies/quantum
\end{FramedParagraphWithFootnotes}


\subsection{Education, and changes in the educational system}
FG-NOTE: I wonder if we need a separation between education of RSEs and changes in the education of the broader society.

\subsection{Impact of AI and automation}
Is it better than cheap PhDs?

\subsection{How have academic Institutions transformed by then? ( Point 4 in the issue )}
- Emerging structures:
  - Universities/Institutes with central pools of RSEs, each specializing in data management, visualization, HPC, etc.
  - Potential for closer collaboration between domain scientists and RSEs to share workload effectively.

- Challenges:
  - Budget and administrative hurdles to creating permanent, well-funded RSE teams.
  - Aligning expectations: researchers might not realize that specialized RSE services exist (and are beneficial).

- Networking:
  - Need to share best practices among institutions with successful RSE teams.
  - Encourage knowledge exchange to find models that suit different organizational constraints.

\subsection{Ethics and social consideration}
FG-NOTE: Given what ChatGPT has put in there, I wonder if we need to separate these two items
MM-NOTE: (Flo believes it was from Michele) Are we assuming that progress will bring prosperity? Will we remain constrained by needing to produce products/artifacts?

- Military or other sensitive applications:
  - RSEs (like physicists, mathematicians, etc.) might face ethical dilemmas if their work can be repurposed for weaponry.
  - Importance of having personal “red lines” and awareness of how software can be used.

- Corporate dominance:
  - AI and coding platforms might be increasingly controlled by a small number of large corporations.
  - RSE community might need to develop or advocate for open-source alternatives.


\subsection{What are new tasks of RTPs?}
FG-NOTE: RSEs are one type of specialization of RTPs ins science. Do we see any other?

\subsection{Impact of Alternate Career Paths for Researchers on the Structure of Research Institutes}
FG-NOTE: By Nick. I've put it here, due to the realtion with the RTP question.


\subsection{Impact of Complexity and Usability in Research Software}

The workshop participants found that in earlier learning stages,
details may be intentionally hidden,
fearing a diminishing capability to handle technical complexity in research environments.
As a result, RSEs invest more effort in making individual research software components
less complex with better user experience.
This made participants worry about growing mismatch of peoples' expectations.
Participants identified the over-reliance on generative AI tools as an extreme manifestation of this issue.
No concrete examples were given on which complexities are now hidden behind abstractions.

We acknowledge but would also like to challenge the observations of the participants.
We believe that ``we are standing on the shoulders of giants'':
By reducing the complexity of individual software components,
we are now able to develop systems of complexity and capability not previously possible,
allowing the scientific community to address questions not previously in grasp.
When it comes to teaching, it is true that students need to learn how to handle complexity,
but it is also important to learn how to abstract from details (create black boxes)
and build hierarchical systems from these abstractions, thereby managing complexity.
It is clear that we need both RSEs that can dive deep into single projects of significant complexity,
as well as RSEs that can manage complex systems of almost black-box components,
and the future of RSE education should aim to prepare students for both roles.


\subsection{What is in this future for RSEs?}
FG-NOTE: I think we can use this part to wrap things a bit up, and describe how this future then looks like.


\section{REALLY Long-term(10+ years) aspects}
Technically this would mean taking everything from the previous section for granted and look
at what becomes now possible.

\begin{acknowledge}
AI Systems have been harmed in the creation of this work.
\end{acknowledge}

% Bibliography with BibTeX
% ========================
\bibliographystyle{eceasst}
\bibliography{bibliography/bibliography,extra}

\end{document}
